% Homework LaTeX Template
% original version by Sayan Chaudhry (sayanc)
% modified by Tan Yan (tanyan)

\documentclass[11pt]{article}

\title{PageRank}
\author{Tan Yan \and Daniel Bae}

% Useful Packages
\usepackage{amsmath}  % Packages to make typesetting math
\usepackage{amssymb}  %   stuff and symbols easy by the
\usepackage{amsthm}   %   American Mathematical Society
\usepackage{textcomp} % Package for some special symbols
\usepackage{fancyhdr} % Package to make pretty headers
\usepackage[latin1]{inputenc}
\usepackage{enumerate}
\usepackage[hang,flushmargin]{footmisc}
\usepackage{amsfonts}
\usepackage{listings}
\usepackage{hyperref} % Package to add image support
\hypersetup{          
    colorlinks=true,  
    linkcolor=red,    
    filecolor=magenta,      
    urlcolor=cyan,
}
\usepackage{graphicx} % Package to add image support
\graphicspath{        
    {assets/}         % Upload all images in assets/
}

% Theorem Styles
\theoremstyle{definition}
\newtheorem{theorem}{Theorem}
\newtheorem{lemma}[theorem]{Lemma}
\newtheorem{corollary}[theorem]{Corollary}
\newtheorem{proposition}[theorem]{Proposition}
\newtheorem{definition}[theorem]{Definition}
\newtheorem{example}[theorem]{Example}
\newtheorem{notation}[theorem]{Notation}

% Page Setup
\oddsidemargin0cm
\topmargin-2cm
\textwidth16.5cm
\textheight23.5cm
\setlength{\parindent}{0pt}
\setlength{\parskip}{5pt plus 1pt}
\setlength{\headsep}{1.7em}
\setlength{\headheight}{30pt}
\renewcommand{\baselinestretch}{1.25}

% Math Commands :: General
\newcommand{\micdrop}{\hfill \qedsymbol}
\newcommand{\abs}[1]{\left| #1\right|}
\newcommand{\floor}[1]{\left\lfloor#1\right\rfloor}
\newcommand{\Lap}[1]{\mathcal{L}\left\{#1\right\}}

% Math Commands :: Proofs 
\newcommand{\wts}{\text{We want to show that }}
\newcommand{\st}{\text{ such that }}
\newcommand{\afsoc}{\text{Assume for the sake of contradiction that }}
\newcommand{\Wlog}{\text{Without loss of generality, assume that }}

% Math Commands :: Sets
\newcommand{\Z}{\mathbb{Z}}
\newcommand{\R}{\mathbb{R}}
\newcommand{\N}{\mathbb{N}}
\newcommand{\Q}{\mathbb{Q}}
\newcommand{\C}{\mathbb{C}}
\newcommand{\U}{\mathcal{U}}
\renewcommand{\emptyset}{\varnothing}

% Math Commands :: Induction
\newcommand{\rhs}{\texttt{RHS}}
\newcommand{\lhs}{\texttt{LHS}}
\newcommand{\bc}{\texttt{Base Case}}
\newcommand{\bcs}{\texttt{Base Cases}}
\newcommand{\ih}{\texttt{Induction Hypothesis}}
\newcommand{\is}{\texttt{Induction Step}}

% Math Commands :: Probability
\newcommand{\p}[1]{\mathbb{P}\left\{#1\right\}}
\newcommand{\e}[1]{\mathbb{E}\left[#1\right]}
\newcommand{\var}[1]{\mathrm{Var}\left[#1\right]}

% Math Commands :: Linear Algebra
\DeclareMathOperator{\Diam}{diam}
\newcommand{\diam}[1]{\Diam\left(#1\right)}
\DeclareMathOperator{\nul}{Null}
\newcommand{\Null}[1]{\nul\left(#1\right)}
\renewcommand{\vec}[1]{\mathbf{#1}}
\newcommand{\threevec}[3]{\left[\begin{array}{r} #1 \\ #2 \\ #3\end{array}\right]}
\newcommand{\fourvec}[4]{\left[\begin{array}{r} #1 \\ #2 \\ #3\\#4\end{array}\right]}
\DeclareMathOperator{\Rank}{rank}
\newcommand{\rank}[1]{\Rank\left(#1\right)}
\DeclareMathOperator{\Sp}{Span}
\renewcommand{\sp}[1]{\Sp\left\{#1\right\}}
\DeclareMathOperator{\col}{Col}
\newcommand{\Col}[1]{\col\left(#1\right)}

\begin{document}
\maketitle
\begin{abstract}
Enter a short summary here. What topic do you want to investigate and why? What experiment did you perform? What were your main results and conclusion?
\end{abstract}

\section{Introduction}

\section{Definitions}
\begin{definition}
    A stochastic matrix, probability matrix, or Markov transition matrix is ...
    A column-stochastic or left-stochastic matrix is ...
\end{definition}

\begin{definition}
    A positive matrix is a matrix with all positive entries
\end{definition}

\begin{definition}
    An Markov transition matrix is said to be irreducible if ...
    A positive stochastic matrix is necessarily irreducible
\end{definition}

\begin{definition}
    Let $G$ be a directed graph of the webpages, with each webpage as a node and links between the webpages as edges ... such that $G_{ij} = 1$ if ...
    also let $n$ denote the number of nodes ...
    let $L(i)$ denote the number of outgoing edges ...
\end{definition}

\begin{definition}
    A dangling node is a node that has no outgoing edges
\end{definition}

\begin{definition}
    Let $P$ be the probablistic matrix 
\end{definition}

\begin{definition}
    Let $d$ denote the ``damping factor'' ... representing the probability of a web surfer randomly traveling from one page to any other page. $0 \leq d < 1$.
\end{definition}

\begin{definition}
    Let $\vec{1}$ 
\end{definition}

\begin{definition}
    Let $M$ denote the ``PageRank Matrix'' or ``Google Matrix'' defined by Page and Brin.
    Define 
\end{definition}

\begin{definition}
    Let $\lambda$ denote an eigenvalue of the square matrix $A$ and let $\vec{v}$ denote its corresponding eigenvector, such that $A \vec{v} = \lambda \vec{v}$.
    A probablistic eigenvector is...
\end{definition}

\section{Main Ideas}

\begin{theorem}
    (The Frobenius Theorem) If 
\end{theorem}

\begin{theorem}
    (Convergence) 
\end{theorem}

This implies that after infinitely many random walks, the web surfer is most likely to be on the webpage corresponding to ...

\begin{lstlisting}[language=python, basicstyle={\small\ttfamily}, numbers=left]
def build_prob_matrix(adj_list):
# number of nodes
n = len(adj_list)
P = np.zeros((n,n), dtype=float)
for (j, connected_nodes) in enumerate(adj_list):
    if connected_nodes: # non-empty
        P[connected_nodes, j] = 1.0 / len(connected_nodes)
    else: # dangling node
        P[:, j] = 1.0 / n
return P
\end{lstlisting}

% \section{Results}

\section{Conclusion}

\begin{thebibliography}{9}
\bibitem{}

\end{thebibliography}
\end{document}


